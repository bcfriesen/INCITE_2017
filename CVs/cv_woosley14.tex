\documentclass[12pt,letterpaper,english]{article}
\usepackage[T1]{fontenc}
\usepackage[pagestyles]{titlesec}
\usepackage{txfonts}
\usepackage{blindtext}
\usepackage{xcolor}
\usepackage{fancyhdr}
\usepackage[top=1in, bottom=1in, left=1in, right=1in] {geometry}
\usepackage{mdwlist}

\pagestyle{fancy}

\raggedright

\begin{document}

\setlength{\parindent}{0in}


\pagestyle{fancy} \lhead{Approaching Exascale Models of Astrophysical Explosions
} \rhead{M.~Zingale} \renewcommand{%
\headrulewidth}{0.0pt}


\centering {{\bf Curriculum Vitae}\\
{\bf Stan E. Woosley}\\
{Department of Astronomy and Astrophysics, University of
  California, Santa Cruz, CA 95064} \\[1mm]
%
{\it phone:}~(831) 459-2976 \hskip 2mm
{\it e-mail:}~woosley@ucolick.org \hskip 2mm\\
{\it web:}~http:/\hspace{-0.25em}/www.supersci.org/ \\
}

\smallskip

\begin{flushleft} {\bf Professional Preparation}
{\parindent 16pt

PhD - 1971 - Rice University - Space Sciences - advisor D. D. Clayton \\  
BA  - 1966 - Rice University - Physics \\ 

}

\vspace{.04in}
{\bf Appointments}
\begin{tabular*}{\textwidth}{p{1.0 in}p{5.3 in}}
\\[-3 mm]
1975 - present              & {\it University of California, Santa Cruz, 
                            Department of Astronomy and Astrophysics:} \\
                            & Assist. Prof. (1975 - 1978); Assoc. Prof. (1978 - 1983); 
                              Full Professor (1983 - 2001); Distinguished Prof. 
                              (2001 - present). 39 years teaching experience. 
                              Department Chair for 10 years.

\\[-3mm]
1972 - 1975             & {\it Kellogg Radiation Laboratory, California Institute 
                              of Technology:} \\
                        & Research Associate - {\it advisor: } Prof. W. A. Fowler\\

\end{tabular*}\\



\vspace{.04in}
{\bf Five Recent Publications Most Relevant to This Proposal}
\vspace{-6pt}
\begin{enumerate*} 
\item {\it Type Ia Supernovae from Merging White Dwarfs. I. Prompt
  Detonations}, R. Moll, C. Raskin, D. Kasen, and S.  Woosley,
  Astrophys. J. 785, 105 - 117, (2014) \\
\item {\it The Deflagration Stage of Chandrasekhar Mass Models for
  Type Ia Supernovae. I. Early Evolution}, C. Malone et al (including
  S. Woosley), Astrophys. J., 782, 11 - 34 (2014)\\
\item {\it Multi-dimensional Models for Double Detonation in
  Sub-Chandrasekhar Mass White Dwarfs}, R. Moll and S. Woosley,
  Astrophys. J, 774, 137 - 151, (2013)\\
\item {\it Sub-Chandrasekhar Mass Models for Supernovae}, 
    S. E. Woosley and D. Kasen, Astrophys. J., 734, 38, (2011) \\ 
\item {\it The Diversity of Type Ia Supernovae from Broken
  Symmetries},D. Kasen, S. Woosley, and F. R\"opke, Nature, 460, 869, 
  (2009) \\ 
\end{enumerate*} 

\vspace{-6pt}
{\bf Research Interests and Expertise}
{\parindent 16pt


Over 40 years experience working on nucleosynthesis theory and models
for supernovae of all types, as well as models for other forms of explosive
astrophysical transients, especially x-ray bursts and gamma-ray
bursts. Originated the currently accepted (collapsar) model for
gamma-ray bursts and, with Ron Taam, the thermonuclear model for x-ray
bursts. In 2005, awarded both the Bethe Prize of the APS and the Rossi
Prize of the AAS for work on supernovae, gamma-ray bursts, and
nucleosynthesis. Work has garnered over 35,000 citations
}

\newpage

\vspace{.04in}
{\bf Synergistic Activities}
\vspace{-6pt}
\begin{enumerate*} 
\item Former PI and Director, SciDAC Computational Astrophysics
Consortium, 2006 - 2012 \\ 
\item Member National Academy of Sciences (2006) and
American Academy of Arts and Sciences (2001) \\ 
\item Consultant in astrophysics and nuclear physics, Lawrence
Livermore National Lab, 1971 - 2012 \\ 
\item  Recipient of major research funding from the DOE, NSF, NASA,
and the UC Lab Management fund for research in supernova models and
nucleosynthesis \\ 
\end{enumerate*} 

\vspace{-6pt} {\bf Collaborators ({\emph{past 5 years including name
      and current institution}})} {\parindent 16pt Andy Aspden (LBNL),
  Ann Almgren (LBNL), John Bell (LBNL), Josh Bloom (UCB), Justin Brown
  (UCSC), Adam Burrows (Princeton), Ken Chen (UCSC), Luc Dessart
  (CNRS), Gary Glatzmaier (UCSC), Shawfeng Dong (UCCS), Gary
  Glatzmaier (UCSC), Rob Hoffman (LLNL), Candace Joggerst (LANL), Dan
  Kasen (UCB), Alan Kerstein (Sandia), Chryssa Kouveliotou (NASA
  Marshall), Elizabeth Lovegrove (UCCS), Chris Malone (LANL), Paulo
  Mazzali (ARI), Rainer Moll (MPA), Andy Nonaka (LBNL), Luke Roberts
  (UCSC), Cody Raskin (UCB), Fritz Roepke (MPA Garching), Tuguldur
  Sukhbold (UCSC), Daniel Whalen (StSci), Ralph Wijers (Amsterdam),
  Sung-Chul Yoon (U. Bonn), Mike Zingale (SUNYSB), Weiqun Zhang (LBNL)
}
\end{flushleft}

\end{document}
