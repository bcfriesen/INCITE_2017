\documentclass[11pt,letterpaper,english]{article}
\usepackage[T1]{fontenc} % Standard package for selecting font encodings
\usepackage{txfonts} % makes spacing between characters space correctly
\usepackage{xcolor} % Driver-independent color extensions for LaTeX and pdfLaTeX.
%\usepackage{blindtext} % To create text
%\usepackage{mdwlist} % mdwlist for compact enumeration/list items 
%\usepackage[pagestyles]{titlesec} % related with sections—namely titles, headers and contents
\usepackage{fancyhdr} % header footer placement

\usepackage[top=1in, bottom=1in, left=1in, right=1in] {geometry} % Margins
\usepackage{graphicx}   % Essential for adding images to you document.

\usepackage{sectsty}
\sectionfont{\large}
\subsectionfont{\normalsize}
\subsubsectionfont{\normalsize \it}

\usepackage{caption}
\captionsetup{labelsep=period}

\usepackage{hyperref}

\pagestyle{fancy} % allows you to use the header and footer commands 

\raggedright
\begin{document}

\setlength{\parindent}{0in} % Amount of indentation at the first line of a paragraph.

\pagestyle{fancy} \lhead{Michael Zingale} \rhead{Lead PI} \renewcommand{%
\headrulewidth}{0.0pt}


\thispagestyle{plain}

\centering {\bf Curriculum Vitae: Adam Jacobs}\\
{Department of Physics and Astronomy, Stony Brook University, Stony Brook, NY 11794-3800} \smallskip
{{\it phone:}~(501) 326-8316 \hskip 2mm
{\it e-mail:}~Adam.Jacobs@stonybrook.edu \hskip 2mm \\[-0.25em]
{\it web:}~\url{http://www.astro.sunysb.edu/amjacobs/}}

\begin{flushleft} {\bf Professional Preparation}
{\parindent 16pt

PhD in Physics with Astronomy Concentration, Stony Brook University (2015, expected)\\ 
BA in Physics, \emph{summa cum laude}, Hendrix College (2009)\\ 
BA in Computer Science, \emph{summa cum laude}, Hendrix College (2009)\\ 
}

\vspace{.04in}
{\bf Appointments}
{\parindent 16pt

2010--present: {\em Research Assistant}, Stony Brook University \\ 
2009--2010: {\em Teaching Assistant}, Stony Brook University \\ 
Summer 2008:  {\em National Undergraduate Fellow}, Princeton Plasma Physics Laboratory \\ 
2006--2008: {\em Research Assistant}, Hendrix College \\
}

\vspace{.04in}
{\bf Publications Most Relevant to This Proposal}
\vspace{-6pt}
\begin{enumerate} \itemsep1pt \parskip0pt \parsep0pt
\item {\it Low Mach Number Modeling of Convection in Helium Shells on
      Sub-Chandrasekhar White Dwarfs. II. Ignition and Convection,} A.~M.~Jacobs,
      M.~Zingale, A.~Nonaka, A.~S.~Almgren, J.~B.~Bell, {\em in prep}.

\item {\it Low Mach Number Modeling of Double-Detonation Type Ia Ignition,} 
      A.~M.~Jacobs, April 11$^{\mathrm{th}}$ 2014, Max Planck Institute for Astrophysics' 
      XVII Workshop on Nuclear Astrophysics, Ringberg Castle Conference Site, Germany.
  
\item {\it Low Mach Number Modeling of Convection in Helium Shells on 
      Sub-Chandrasekhar Mass White Dwarfs,} A.~M.~Jacobs, M.~Zingale, A.~Almgren, J.~Bell, A.~Nonaka,
      S.~Woosley, May 13$^{\mathrm{th}}$ - 17$^{\mathrm{th}}$ 2013, Fifty-One Erg International 
      Workshop, Raleigh, North Carolina.

\end{enumerate} 

\vspace{-6pt}
{\bf Research Interests and Expertise}
{\parindent 16pt

  My research is driven by a fascination with the intersection of
  high performance computing, numerical analysis, and nuclear astrophysics.  
  
  Numerically, I work to push the methods of computational hydrodynamics
  to scale.  Doing so requires the development of algorithms and techniques
  for effectively modeling larger problems and reducing communication costs.
  My main effort now is to target nuclear reaction integrations at GPGPUs
  like those in the Titan supercomputer.
  
  Astrophysically, I work to understand the Universe's violent thermonuclear
  explosions associated with observations of transient events.  Currently,
  I am investigating a promising progenitor model for Type Ia supernovae:
  sub-Chandrasekhar double-detonations.  I am modeling the pre-explosive
  convective phase and ignition in 3D using the low Mach number simulation
  code Maestro.

}

\vspace{.04in}
{\bf Synergistic Activities}
\vspace{-6pt}
\begin{enumerate} \itemsep1pt \parskip0pt \parsep0pt
\item Contributing developer of the publicly-available low Mach number
  hydrodynamics code Maestro,
  \url{http://bender.astro.sunysb.edu/Maestro/} \\

\item Contributing developer of the publicly-available adaptive mesh
  refinement framework BoxLib,
  \url{https://ccse.lbl.gov/BoxLib/} \\


\item Participated in the 2013 
  \href{http://extremecomputingtraining.anl.gov/}{Argonne Training Program on Extreme Scaling Computing} and the 
  2013 \href{https://www.xsede.org/web/summerschool13/home}{International HPC Summer School}. \\

\item Volunteer speaker at Greene Correctional Facility, Coxsackie, New York, 
  presenting a talk entitled
  {\em Cosmology: Precisely Calculating How Little We Know}. \\

\item Mentor in the Adopt-a-Physicist program, connecting high school physics 
  students across the United States with physics graduates. \\

\item Co-administrator of the Diversity in Physics and Astronomy Facebook page,
  boasting over 1000 members. \\

\end{enumerate} 

\vspace{-6pt}
{\bf Collaborators ({\emph{past 5 years including name and current institution}})} 
{\parindent 16pt

Ann Almgren, Lawrence Berkeley National Lab\\
Alan Calder, Stony Brook University\\
Matthew Emmett, Lawrence Berkeley National Lab\\
Mike Lijewski, Lawrence Berkeley National Lab\\
Chris Malone, Los Alamos National Lab\\
Bronson Messer, Oak Ridge National Lab, University of Tennessee-Knoxville\\
Andy Nonaka, Lawrence Berkeley National Lab\\
Stan Woosley, University of California-Santa Cruz

}


\end{flushleft}

\end{document}
