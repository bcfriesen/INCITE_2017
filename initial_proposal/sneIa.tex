Type Ia supernovae (SNe Ia) are the thermonuclear explosion of a
carbon/oxygen white dwarf (WD) in a binary system.  These bright
explosions rival the luminosity of the host galaxy, making them
visible at cosmological distances. Furthermore, the
brightest events take the longest to dim, which allows their use as
distance indicators and led to discovering the accelerating 
expansion of the Universe, a remarkable advance.

A fundamental uncertainty in SNe Ia is the nature
of the progenitor. Is it a single WD accreting from a normal companion star
(single degenerate scenario) or two WDs that inspiral (or
violently collide) and merge (the double degenerate scenario)? (For a
review of explosion models, see \cite{calder:2013}).  No conclusive 
observation of the progenitor system exists, but in all cases 
the majority of the carbon and oxygen in the WD(s) is converted into
iron/nickel and intermediate-mass elements like silicon, and this
nuclear energy release unbinds the star.
For many years, the near-Chandrasekhar-mass single degenerate model
(henceforth the Chandra model) saw the most attention.  The
Chandrasekhar mass is the maximum possible mass of a WD, so
explosions near this limit imply SNe Ia would be alike.  Contemporary
observations suggest diversity and it is not clear if nature makes SNe
Ia this way---massive carbon/oxygen WDs in binary systems are rare.  
Accordingly, our proposed research explores other explosion models,
focusing on merging WDs, sub-Chandra explosion models, and the convective 
Urca process thought to occur during the late stage of accretion in the
Chandra model.  INCITE provides us the opportunity to make fundamental
contributions to each of these problems.

The standard picture for mergers has the WDs
inspiraling as gravitational radiation removes orbital energy.
The less massive star will become tidally disrupted and the more
massive star will accrete this material.  A longstanding concern is
that when this mass transfer begins, thermonuclear burning can ignite
at the edge of the star, converting it to O/Ne/Mg, and leading to the
collapse of the WD into a neutron star, instead of an SNe Ia.
% ~\cite{saionomoto:2004,fryerdiehl:2008} 
This system is inherently 3D, and only through detailed simulation
can we understand the dynamics of the mass transfer, and thereby
assess the feasibility of this model.

Sub-Chandra models
% ~\cite{fink:2010,shen:2010,sim:2012} 
have the advantage that systems with a moderate mass WD
(0.8--1.0 solar masses) are known and abundant, and this model also
can reproduce the known delay-time distributions and SNe Ia
rates---something the Chandra model struggles with~\cite{ruiter:2011}.
In this model, the WD accretes a layer of He on its surface
that detonates, sending a compression wave into the underlying C/O WD
that subsequently ignites a detonation in the core that unbinds the
star.  An early problem with this model was whether it 
could produce abundances and spectra compatible with observations, 
%\cite{hoeflich:1996,nugent:1997,kromer:2010}.
but the modern view suggests that a detonation in a very thin He layer
can trigger the explosion without over-producing surface iron-group
elements~\cite{fink:2010}.  Our goal is to understand under what
conditions ignition can take place in these thin shells when a
realistic 3D convective field is realized, as well as the character
and subsequent evolution of that ignition.
%---this is what we will answer.

The final piece of the Ia puzzle we will explore is the the earliest
stage of the convection in the WD as it approaches the Chandrasekhar
limit in the Chandra model, when neutrino losses in the reaction can
alter the dynamics of the convection (called the URCA process).  The
only multi-dimensional simulations of URCA~\cite{URCA} to date have
been 2D, with only a portion of the star modeled.  Applying the
low-Mach code \maestro\ to this problem will allow simulations with
unprecedented realism.  These calculations will build upon the 
work of ignition in the Chandra model that we performed under
previous INCITE awards~\cite{Non12}.
