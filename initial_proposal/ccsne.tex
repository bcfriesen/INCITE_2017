Core-collapse SNe (ccSNe) mark the death of massive ($M \gtrsim 10
M_{\odot}$ [solar masses]) stars, yielding bright and energetic
explosions and the births of neutron stars and black holes.  After
several million years of evolution, a massive star's core is composed
of iron (and similar `iron-peak' elements) from which no further
nuclear energy can be released by fission or fusion.  Outside the
Fe-core are shells representative of previous burning stages--- a
silicon shell, oxygen shell, etc., out to a helium shell surrounded by
an envelope of hydrogen.  At the base of the Si-shell, nuclear burning
continues, growing the Fe-core below.  When the mass of the Fe-core
reaches the limiting Chandrasekhar mass, it starts to collapse.
During the collapse, the inner core will become opaque to neutrinos
and surpass the density of atomic nuclei
($\gtrsim$$2.5\times10^{14}\,\gcc$) reaching densities where
individual nuclei merge together into nuclear matter.  Above nuclear
density, the nuclear equation of state (EoS) stiffens and the core
rebounds like an over-compressed spring, launching a rebound, or
bounce, shock from the newly formed neutron star (a proto-NS).  The
rebound shock progresses outward through the rest of the infalling
core, heating and dissociating the infalling nuclei to free nucleons
and radiating a large burst of neutrinos. Thermal energy removed from
the shocked material by neutrinos and nuclear dissociation halts the
progress of the shock rendering it a standing accretion shock (SAS)
with a radius of 100--200~km about 50~ms after it was
launched. Neutrino emission from the cooling proto-neutron star (PNS),
aided by multi-D effects (e.g.\ convection and the stationary
accretion shock instability) have been invoked to `revive' the shock
and continue its progress to the stellar surface.  Our group and
others have recently obtained realistic explosions in
3D simulations \cite{lentzetal2015,melson}, but only a
narrow slice of the full space of progenitor mass, metallicity,
rotation rates, and initial magnetic fields has been explored to
date. We propose to begin to extend this coverage through a 3D
simulation of a light (i.e.\ $\lesssim 12 M_{\odot}$) progenitor, paying
special attention to the nuclear burning that can still occur during
the collapse of such a low-mass star.
