\documentclass[11pt,letterpaper,english]{article}
\usepackage[T1]{fontenc} % Standard package for selecting font encodingsamely titles, headers and contents
\usepackage{txfonts} % makes spacing between characters space correctly
\usepackage{xcolor} % Driver-independent color extensions for LaTeX and pdfLaTeX.
%\usepackage[pagestyles,raggedright]{titlesec} % related with sections—n
%\usepackage{blindtext} % To create text
\usepackage{fancyhdr} % header footer placement

\usepackage[top=1in, bottom=1in, left=1in, right=1in] {geometry} % Margins
\usepackage{graphicx}  % Essential for adding images to you document.

\usepackage{sectsty}
\sectionfont{\normalsize}
\subsectionfont{\normalsize}
\subsubsectionfont{\normalsize \it}

\usepackage{caption}
\captionsetup{labelsep=period}

\input newcommands

\pagestyle{fancy} % allows you to use the header and footer commands

%\raggedright
\begin{document}

%\setlength{\parindent}{0in} % Amount of indentation at the first line of a paragraph.

\pagestyle{fancy} \lhead{Approaching Exascale Models of Astrophysical Explosions} \rhead{PI: Zingale} \renewcommand{%
\headrulewidth}{0.0pt}

\begin{center}
\bf {PROJECT EXECUTIVE SUMMARY} 
\end{center}
%
\begin{flushleft}
\textbf{Title}: Approaching Exascale Models of Astrophysical Explosions %\smallskip

\textbf{PI and Co-PI(s)}: M.~Zingale (PI); A.~S.~Almgren, J.~B.~Bell, A.~C.~Calder, 
B.~Friesen, R.~Hix, A.~Jacobs, D.~Kasen, M.~P.~Katz, E.~Lentz, C.~M.~Malone, B.~Messer, T.~Papatheodore, \& W.~Zhang %\smallskip

\textbf{Applying Institution/Organization}: Stony Brook University %\smallskip

\textbf{Number of Processor Hours Requested}: 80 Mh (year 1), 111 Mh (year 2), 122 Mh (year 3) %\smallskip

\textbf{Amount of Storage Requested}: 800 TB (year 1), 900 TB (year 2), 1 PB (year 3) \MarginPar{needs updating}

\end{flushleft}
%
%% The executive summary should accurately describe the proposed research
%% and the high-impact scientific or technical advances you will realize
%% with the proposed INCITE allocation. Industry organizations should
%% also summarize the potential economic or strategic business impact of
%% the proposed research.\\
%
\vskip -0.5em

\noindent {\bf Executive Summary:} We propose a comprehensive study of
stellar explosions using a suite of state-of-the-art application
codes.  We will explore multiple progenitor models of Type Ia
supernovae (SNe Ia), the physics of X-ray bursts (XRBs), the radiative
ablation in black widow pulsars (BWP), and core-collapse supernovae.
Our code suite already runs on titan at OLCF, and through this
proposal, we will unify our efforts at utilizing the GPUs for
accelerating the microphysics.

Our simulations codes, \maestro\ and \castro, were designed
specifically for the efficient modeling of astrophysical explosions.
These codes will be used for the convection studies of the sub-Chandra
model of SNe Ia, white dwarf mergers, and the convective Urca process,
all of the XRB studies, and the studies of the BWP systems.  These
codes make excellent use of the multi-core architecture by using a
hybrid approach to parallelism (MPI and OpenMP), and we have made
considerable progress in targeting the effective use of GPUs.  We will
build on the already strong parallel performance of these codes by
implementing new efficient algorithms for core solvers, building on
our close collaboration with the \boxlib\ group at LBNL.  Our
collaboration has been together more than 10 years and includes core
developers of the simulation codes along with the scientific expertise
to analyze the simulations.  Finally, both codes are available to the
public, and performance (or physics) improvements developed in the
course of the proposed work will be made available to the community.

The ccSNe simulations will utilitize \chimera\ \MarginPar{Bronson add a few sentences here}

Finally, some of the explosion calculations for sub-Chandra SNe Ia
will utilitze \flash.

A common theme in our simulations is the expense of the microphysics,
in particular nuclear reaction networks.  All of our calculations
require accurate models of the energy deposition from reactions, and a
focus in this proposed period will be to utilize larger networks.
This requires offloading the expensive rate operations to the GPU.  We
have several efforts along this way already, and through this
proposal, we will merge our efforts to produce a set of GPU-enabled
community solvers for microphysics.

\end{document}
