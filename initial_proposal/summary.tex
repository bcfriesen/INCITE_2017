\documentclass[11pt,letterpaper,english]{article}
\usepackage[T1]{fontenc} % Standard package for selecting font encodingsamely titles, headers and contents
\usepackage{txfonts} % makes spacing between characters space correctly
\usepackage{xcolor} % Driver-independent color extensions for LaTeX and pdfLaTeX.
%\usepackage[pagestyles,raggedright]{titlesec} % related with sections—n
%\usepackage{blindtext} % To create text
\usepackage{fancyhdr} % header footer placement

\usepackage[top=1in, bottom=1in, left=1in, right=1in] {geometry} % Margins
\usepackage{graphicx}  % Essential for adding images to you document.

\usepackage{sectsty}
\sectionfont{\normalsize}
\subsectionfont{\normalsize}
\subsubsectionfont{\normalsize \it}

\usepackage{caption}
\captionsetup{labelsep=period}

\input newcommands

\pagestyle{fancy} % allows you to use the header and footer commands

%\raggedright
\begin{document}

%\setlength{\parindent}{0in} % Amount of indentation at the first line of a paragraph.

\pagestyle{fancy} \lhead{Approaching Exascale Models of Astrophysical Explosions} \rhead{PI: Zingale} \renewcommand{%
\headrulewidth}{0.0pt}

\begin{center}
\bf {PROJECT EXECUTIVE SUMMARY} 
\end{center}
%
\begin{flushleft}
\textbf{Title}: Approaching Exascale Models of Astrophysical Explosions %\smallskip

\textbf{PI and Co-PI(s)}: M.~Zingale (PI); A.~S.~Almgren, J.~B.~Bell, A.~C.~Calder, 
A.~Jacobs, D.~Kasen, M.~P.~Katz, C.~M.~Malone, \& S.~E.~Woosley %\smallskip

\textbf{Applying Institution/Organization}: Stony Brook University %\smallskip

\textbf{Number of Processor Hours Requested}: 80 Mh (year 1), 111 Mh (year 2), 122 Mh (year 3) %\smallskip

\textbf{Amount of Storage Requested}: 800 TB (year 1), 900 TB (year 2), 1 PB (year 3) %\smallskip

\end{flushleft}
%
%% The executive summary should accurately describe the proposed research
%% and the high-impact scientific or technical advances you will realize
%% with the proposed INCITE allocation. Industry organizations should
%% also summarize the potential economic or strategic business impact of
%% the proposed research.\\
%
\vskip -0.5em

\noindent {\bf Executive Summary:} We propose to carry out a 
comprehensive study of two classes of thermonuclear-powered stellar
explosions involving compact objects, Type Ia supernovae (SNe Ia) and
X-ray bursts (XRBs), using the state-of-the-art multiphysics simulation codes
\maestro\ and \castro.  This work builds upon the successes of our
current INCITE campaign while shifting the focus to exciting new problems.

SNe Ia are the thermonuclear explosion of a white dwarf (a compact
stellar remnant up to 1.4 times more massive than the Sun but the size
of the Earth). There are several progenitor systems proposed to meet 
the observational constraints
on SNe Ia: the Chandrasekhar-mass white dwarf model; the
sub-Chandrasekhar mass white dwarf model (sub-Ch); and the 
merger of two white dwarfs. In the three years of this proposal, 
we will focus on the latter two models, complementing 
the study of the Chandrasekhar-mass model undertaken during our 
current INCITE award.  This shift in focus mirrors that of the field, 
responding to increasingly abundant observations showing the diversity 
of SNe Ia explosions.

The sub-Ch SNe Ia model involves the ignition of a surface
helium layer and subsequent compression and explosion of the
underlying white dwarf.  Many open questions remain about the
viability of this model. If too much surface He is ignited,
then iron-group elements will be over produced at the surface,
and the resulting explosion will not look like an SNe Ia.  We also don't
know under what conditions and in how many locations the ignition of the
He begins. We will address these questions directly by modeling the
final stages of convection before the explosion, the ignition, and the 
subsequent explosion. The unique capabilities of our two simulation codes,
described below, allow us to seamlessly model these
three stages of the event.

We will also consider the inspiral and merger of two white dwarfs
as a model for SNe Ia.  There is a lot of potential variability in this
model, since the white dwarf masses are likely unequal.  Our initial
goal will be to explore a
variety of initial configurations and understand the dynamics that
ensues when the stars first make contact.
We will determine whether the conditions that arise lead to an immediate
detonation or whether the merged remnant needs to relax before the
explosion occurs.

XRBs occur on the surfaces of neutron stars, compact
stars more extreme than white dwarfs---more mass than
the Sun is packed into a star just 10~km in radius.  The extreme
gravity at the surface compresses H/He fuel 
accreted from a companion to the point of thermonuclear runaway.  
The resulting X-ray burst releases a large amount of energy
as H and He burn to heavier elements.
These are repeating events, making them excellent probes of 
neutron stars.  Most theoretical understanding of XRBs comes from
1D, spherically-symmetric models, but
as convection dominates the onset of the burning, a proper treatment
requires 3D simulation.  We will perform the most
detailed simulations of convective burning in XRBs, yielding a detailed
picture of how the convection alters the nucleosynthesis and observables.

Our simulations codes, \maestro\ and \castro, were designed
specifically for the efficient modeling of astrophysical explosions.
They are running on titan---the target of this proposal.
The codes make excellent use of the multi-core architecture by using 
a hybrid approach to parallelism (MPI and OpenMP), and we
have made considerable progress in targeting the effective use of GPUs. 
We will build on the already strong parallel performance of these codes
by implementing new efficient algorithms for core solvers, building on our close collaboration 
with the \boxlib\ group at
LBNL.  Our collaboration has been together more than 10 years
and includes core developers of the simulation codes along with
the scientific expertise to analyze the simulations.  
Finally, both codes are available to the public, and performance (or
physics) improvements developed in the course of the
proposed work will be made
available to the community.



\end{document}
