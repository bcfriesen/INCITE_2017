\documentclass[11pt,letterpaper,english]{article}
\usepackage[T1]{fontenc} % Standard package for selecting font encodings
\usepackage{txfonts} % makes spacing between characters space correctly
\usepackage{xcolor} % Driver-independent color extensions for LaTeX and pdfLaTeX.
\usepackage{hyperref}  %The ability to create hyperlinks within the document

\usepackage{fancyhdr} % header footer placement

\usepackage[top=1in, bottom=1in, left=1in, right=1in] {geometry} % Margins
\usepackage{graphicx}   % Essential for adding images to you document.

\usepackage[numbers]{natbib}

%\usepackage{sectsty}
%% \sectionfont{\normalsize}
%% \subsectionfont{\normalsize}
%\subsubsectionfont{\normalsize \it}

\usepackage[small,compact]{titlesec}
\titlespacing{\section}{0pt}{0.5em}{0.25em}
\titlespacing{\subsection}{0pt}{0.5em}{0.125em}
\titlespacing{\subsubsection}{0pt}{0.5em}{0.25em}
\titlespacing{\paragraph}{0pt}{0.75em}{0.25em}

\setlength{\textfloatsep}{7pt}

\titleformat*{\subsubsection}{\itshape}

\let\oldthebibliography=\thebibliography
  \let\endoldthebibliography=\endthebibliography
  \renewenvironment{thebibliography}[1]{%
    \begin{oldthebibliography}{#1}%
      \setlength{\parskip}{0.1ex}% 
      \setlength{\itemsep}{0.0ex}% 
  }%
  {% 
    \end{oldthebibliography}% 
  }


% commas between multiple footnotes
\newcommand\fnsep{\textsuperscript{,}}

\usepackage{caption}
\captionsetup{labelsep=period}

\setlength{\parskip}{0.1\baselineskip}%
%\setlength{\parindent}{0pt}%

\input newcommands

%\raggedright

\begin{document}

\pagestyle{fancy} 
\lhead{Approaching Exascale Models of Astrophysical Explosions} 
\rhead{PI: Zingale} \renewcommand{%
\headrulewidth}{0.0pt}

\begin{center}
{\bf PROJECT NARRATIVE}
\end{center}


%-----------------------------------------------------------------------------
\section{SIGNIFICANCE OF RESEARCH}

%% Explain what advances you expect to be enabled by an INCITE award that
%% justifies an allocation of petascale resources (e.g., anticipated
%% impact on community paradigms, valuable insights into or solving a
%% long-standing challenge, etc). Place the proposed research in the
%% context of competing work in your discipline or business. The
%% information should be sufficient for peer review in your area of
%% research and also appropriate for general scientific review comparing
%% your proposal with proposals in other disciplines. Potential impact is
%% the predominant determinant for awards. This factor will be assessed
%% by a peer-review panel. Also list any previous INCITE award(s)
%% received and discuss the relationship to the work proposed. {\bf This
%%   section is typically about 4 pages.}

We propose a collaborative investigation of multiple types of stellar
explosions and their precursors using a suite of state-of-the-art
hydrodynamics codes.  Several progenitor models of Type Ia supernovae
will be explored, including their pre-explosion and explosive phases,
using our codes \maestro, \castro, and \flash.  X-ray bursts will also
be studied in their pre-explosive and explosive phases, using
\maestro\ and \castro.  The curious radiation-dominated systems like
the black widow pulsar will be explored with \castro, and
core-collapse supernovae will be studied using \chimera.  All
calculations will be three-dimensional and face the challenges of
capturing the effects of turbulence, instabilities, strong
gravitational interactions, nuclear reactions, and radiation.  These
challenges make these problems INCITE-class, and only the resouces
titan can provide will enable use to further our understanding.
Despite the broad suite of codes, there are common links, namely in
the microphysics (reactions, equations of state) and in our shared
approach to utilitizing the GPUs on titan.  This collaboration is an
expansion of our existing INCITE proposal, but we believe that it
represents the most efficient use of resources, allowing us to work
together as a team to create portable solvers to effectively make
use of this architecture and its successor.  These GPU-enabled
solvers will be made freely available.

Our publication record demonstrates that we have made productive use
of our current INCITE award, and we expect similar productivity
carrying forward.  Furthermore, this INCITE time will be used to train
the next generation of computational scientists---graduate students
feature prominently in the proposed work plan.


\subsection{Type Ia supernovae}

\input sneIa.tex


\subsection{Type I X-ray bursts}

\input xrb.tex


\subsection{Core-collapse supernovae}

\input ccsne.tex


\subsection{Black-widow pulsars}

\input bwp.tex


\subsection{\maestro\ and \castro}

The workhorses of the proposed work will be our state-of-the-art
simulation codes \maestro~\cite{multilevel} and
\castro~\cite{castro:I}, developed over the past 8 years in
collaboration with Lawrence Berkeley Lab.  \maestro\ is tuned to
efficiently model the highly subsonic convective flows that often
precede stellar explosions.  It accomplishes this by reformulating
the equations of hydrodynamics, filtering the
soundwaves from the equations of hydrodynamics, while keeping 
the compressibility effects due to stratification and local heat release.
\maestro\ has been used successfully to model the convection leading
to ignition in the Chandra model for SNe Ia (a focus of our current
INCITE allocation) and will continue to be used for the XRB and sub-Ch
convection simulations proposed here.  \castro\ solves the fully
compressible equations of hydrodynamics, allowing it to model shocks
and explosive phenomena.  It will be the simulation code for the white
dwarf merger simulations, and in later years, for the explosions
following the convection phases modeled by \maestro.  \maestro\ and
\castro\ share the underlying microphysics e.g., equation of state
(EOS) and nuclear reaction networks, as well as the underlying
\boxlib\ library that manages their adaptive mesh refinement (AMR)
grid hierarchy.  This makes it straightforward to transition a problem
from \maestro\ to \castro, as was done during our current INCITE
allocation when studying flame propagation in Chandra model SNe
Ia~\cite{Mal14}.  Both codes are already up and running on our target
platform, titan (OLCF).  Additionally, both codes are publicly
available%
\footnote{\maestro: \url{http://bender.astro.sunysb.edu/Maestro/}}\fnsep\footnote{\castro: \url{https://ccse.lbl.gov/Downloads/downloadCASTRO.html}}---any
performance or physics improvements developed under this INCITE award
will become part of the public releases, benefiting the community
at large.

\section{\flash}

\section{\chimera}


\subsection{Work under previous INCITE awards}

We are currently in the last year of a 2-year INCITE award (AST106;
PI: Zingale; allocation on OLCF/titan) that focused on XRBs and SNe
Ia.  Many of the current investigators have also collaborated on other
previous INCITE awards.  \MarginPar{number of papers}

{\bf \maestro\ convection models}:
%
We completed our initial study of sub-Ch SNe Ia
simulations~\cite{Zin13} with \maestro, and are currently writing up
the results of our second study (led by graduate student co-I Jacobs).
These are the only models in the world that have explored the dynamics
of the convection in the accreted helium layer in a full-3D resolved
simulation.  This study showed that \maestro\ presents a robust
simulation platform for investigating the physics of ignition in the
accreted helium layer.  Follow-on calculations have explored a range
of WD and He-shell masses and what minimum mass is necessary for
ignition.  A paper describing these results is in preparation.

An original focus of our INCITE award was modeling the convection in
Chandra mass white dwarfs leading up to an SNe Ia.  This has been
completed and published in a series of
papers in the Astrophysical Journal~\cite{Zingale:2011,Non12}.  We
used AMR to greatly increase the resolution of
our ignition simulations---this allowed us to explore the resolution
dependence of our results.  We demonstrated that the ignition is most
likely to take place off-center, around 50~km from the center of the
white dwarf, and by following hotspots as they develop in the flow, we
showed that ignition is likely only at a single point.  We
characterized the turbulence in the convective region and showed that
it follows Kolmogorov statistics.  Further, we measured its intensity
and integral length scale and argued that it is unlikely to affect the
flame propagation.

Finally, we did the first simulation using \maestro\ convection
results for a Chandra model SN Ia as initial conditions in \castro\ to
model the subsequent explosion~\cite{Mal14}.  This ``end-to-end''
capability is facilitated by the fact that the two codes share the
same underlying \boxlib\ library.  This is the first 3D calculation of the 
explosion phase in the 
literature to begin with a self-consistent convective field.  This capability added to our understanding
of how the convection affects the evolution of the explosion.  One of
the results from this study was that the flame is unhindered by
turbulent convection for typical ignition locations ($\sim 40$ km
off-center), but is strongly distorted for more centrally-located
ignition.  Even a perfect sphere ignited at the center of the star
will be carried off-center by the turbulent convection.  This results
in asymmetric explosions for all Chandra models, which should be
accounted for when matching to observations.

{\bf \castro\ WD merger models}:
%
A final focus of research under our current INCITE award, begun only
recently, are the initial simulations of the WDWD problem.  \castro\ had not yet been deployed for problems of this type,
and therefore required a number of algorithmic
improvements, including new boundary conditions for the gravity solver
and better treatment of the temperature in the hydrodynamics module.
Titan was used for verification purposes by running simulations of the
early part of a merger scenario, when the two WDs are orbiting each
other at a large distance. In this case, the motion should remain
stable, and as with all hydrodynamics codes, conservation of energy
and angular momentum was paramount. We performed several large-scale
simulations of these, checking various parts of the parameter space to
ensure that we can robustly simulate these events. This helped us fix
a number of underlying issues, and we are beginning production runs.
We note that these calculations are different than those lead by the
UCSC/UCB group above---they started with a merged remnant and then
modeled the subsequent explosion.  In the work proposed here, we 
will model the merger process itself.  This project is led by graduate
student co-I Katz.


\subsection{Significance of our proposed work}

To advance the state of the knowledge of SNe Ia and XRBs, we will
carry out the following sets of simulations (all in 3D):
\begin{tightitem}
\item Full star \maestro\ sub-Ch models with 
  detailed nucleosynthesis and realistic initial models.
\item \castro\ sub-Ch explosion models beginning with the
  \maestro\ initial conditions.
\item The largest-domain-to-date 3D resolved \maestro\ XRB convection
  calculations.
\item The first XRB convection calculations with a burning sub-grid
  model.
\item Highly-resolved \castro\ WDWD models capturing multiple orbits
  and the inspiral.
\end{tightitem}

We are the only group in the world modeling the detailed convection
with realistic nuclear physics in XRBs and the sub-Ch SNe Ia.
In all simulations we will push the size of the nuclear reaction networks
to accurately capture the nucleosynthesis (as discussed later, GPUs will
be used for this task).

XRBs are important probes of the nuclear equation of
state~\cite{ozel:2010,steiner:2010}, and the nucleosynthesis that
takes place will involve the nuclei that are a target of the DOE FRIB
experiment.  Our simulations will advance our understanding on the
burning dynamics and tell us (1) whether any burning products can be
carried to the photosphere, altering the interpretation of
observations; (2) how the full 3D treatment of convection modifies the
nucleosythesis, allowing us to provide feedback to the 1D modelers;
(3) how to create a sub-grid model to enable larger scale simulations;
and (4) ultimately, in year 3 when we are able to implement a sub-grid
model, start to learn how turbulence in the burning alters the
spreading of the front across the neutron star.  With \maestro, we are
in the unique position to address each of these points.

While several groups are investigating explosions in the sub-Ch model,
we are the only group that is modeling the 3D convective field and
ignition that precedes the explosion.  Much like our previous INCITE
work in carrying \maestro\ convection calculations of the Chandra
model into the explosion phase with \castro, our similar work here
(proposed for year 3) for the sub-Ch model will be unmatched.
Initially we will focus on using more realistic models of WDs
generated from stellar evolution codes (we currently use simple
parametrized models), and we will answer the question of which
configurations (WD mass and He mass) lead to ignitions as well as the
timing and geometry of this ignition.  Along the way we will switch
from our simplified 4-isotope network to a more general in-situ
network to better understand the sensitivity of our results to the
nuclear physics, metallicity, and trace abundances of other species.
In addition, we will complete our implementation of nucleosynthetic
post-processing using Lagrangian tracer particles to allow offline
exploration.  This is key, as the exact composition of the helium
layer at detonation can have profound effects on the subsequent
observations, impacting the model's viability as a SNe Ia
progenitor~\cite{kromer:2010}.  All simulations will be 3D, and the
vast majority will model the entire star.  These results will provide
the foundation for \castro\ simulations capable of determining if the
ignition evolves as a detonation and that detonation's subsequent
evolution with realistic 3D initial conditions.
 
Several groups are pursuing WDWD
mergers~\cite{yoon:2007,motl:2007,loren-aguilar:2009,shenetal+11} or
collisions~\cite{raskinetal+10,loren-aguilar:2010,rosswog:2009}. The
majority of these have used smoothed particle hydrodynamics (SPH), a
gridless alternative to the methods we use here.  SPH is known to have
trouble capturing instabilities and has low resolution in regions of
low density---precisely the regions where the stars make contact.  Our
grid--based simulations will provide an important counterpart to
existing simulations.  With the power of AMR, we can push far beyond
the resolutions of the grid-based simulations in the current
literature to levels necessary to assess whether an explosion upon
contact is feasible. \castro\ is ready for these simulations---we have
made the necessary changes for isolated gravity boundary conditions,
and while we will start with simple WD models on the grid, we are
nearly complete in implementing a self-consistent method to initialize a binary
pair of WDs on our grid using the full stellar equation of state.



%-----------------------------------------------------------------------------
\section{RESEARCH OBJECTIVES AND MILESTONES }  

%% Describe the proposed research, including its goals, milestones and
%% the theoretical and computational methods it employs. Goals and
%% milestones should articulate simulation and developmental objectives
%% and be sufficiently detailed to assess the progress of the project for
%% each year of any allocation granted. Milestones should correlate with
%% those in the milestone table. It is especially important that you
%% provide clear connections between the project's overarching
%% milestones, the planned production simulations, and the compute time
%% expected to be required for these simulations (e.g., should correlate
%% with those in the ``Use of Resources Requested'' section below). {\bf
%%   This section is typically about 6 pages.}

\input objectives


%-----------------------------------------------------------------------------
\input readiness
\bibliographystyle{unsrtnat}
\bibliography{refs}




\end{document}
