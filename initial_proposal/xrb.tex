Type I X-ray bursts (XRBs) are the thermonuclear runaway of a thin
layer of hydrogen and/or helium on the surface of a neutron star (NS).
This fuel layer accretes from a binary companion star and the immense
gravitational acceleration on the surface of the NS compresses it to
the point of explosion.  1D hydrodynamic studies have been able to
reproduce many of the observable features of XRBs such as burst
energies ($\sim 10^{39}$ erg), rise times (seconds), durations ($10$'s
-- $100$'s of seconds) and recurrence times (hours to days)
(see \cite{STRO_BILD06} for an overview of XRBs).  By construction,
however, 1D models assume that the fuel is burned uniformly over the
surface of the star, but observations by the {\em Rossi X-ray Timing
Explorer} satellite show coherent oscillations in the lightcurves of
$\gtrsim 20$ outbursts from low-mass X-ray binary systems that suggest
localized ignition and a burning front spreading across the surface of
the NS, modulated by the fast rotation.

Before the actual outburst, the burning at the base of the ignition
column will drive convection throughout the overlying layers and set
the state of the material in which the burning front will propagate.
1D simulations of XRBs usually attempt to parametrize the convective
overturn and mixing using astrophysical mixing-length theory or
through various diffusive processes.  A proper treatment of the
convection in these extreme conditions, free from parameterizations,
requires 3D simulations.  In our current INCITE allocation, we
performed the first ever calculation of convective with resolved
burning in a mixed H/He layer on a NS surface~\cite{xrb-3d}.  3D is
expensive, restricting us to small domains to resolve the burning
processes and small reaction networks to fit the model in memory.  In
the proposed work, we will build upon these first calculations,
pushing to bigger domains and more realistic physics---the computing
power for this can only be provided through INCITE.

Recently, other groups have begun modeling the spreading of a flame
through a NS atmosphere in 2D, using a simplified hydrodynamics code
that only solves the dynamics in the lateral direction and enforces
hydrostatic equilibrium vertically~\cite{cavecchi:2013}.  This allows
for wide-aspect ratio zones that enable a large timestep.  However,
this cannot resolve the dynamics at the burning front, in particular
the effects of turbulence on the flame propagation.  We will perform
our own studies of the lateral flame propagation with full
hydrodynamics in the proposed period.

