\documentclass[11pt,letterpaper,english]{article}
\usepackage[T1]{fontenc} % Standard package for selecting font encodings
\usepackage{txfonts} % makes spacing between characters space correctly
\usepackage{xcolor} % Driver-independent color extensions for LaTeX and pdfLaTeX.
\usepackage{hyperref}  %The ability to create hyperlinks within the document

\usepackage{fancyhdr} % header footer placement

\usepackage[top=1in, bottom=1in, left=1in, right=1in] {geometry} % Margins
\usepackage{graphicx}   % Essential for adding images to you document.

\usepackage[numbers]{natbib}

%\usepackage{sectsty}
%% \sectionfont{\normalsize}
%% \subsectionfont{\normalsize}
%\subsubsectionfont{\normalsize \it}

\usepackage[small,compact]{titlesec}
\titlespacing{\subsubsection}{0pt}{0.5em}{0.25em}
\titlespacing{\subsection}{0pt}{0.5em}{0.5em}
\titlespacing{\section}{0pt}{0.5em}{0.5em}
\titlespacing{\paragraph}{0pt}{0.75em}{0.25em}

\titleformat*{\subsubsection}{\itshape}


\let\oldthebibliography=\thebibliography
  \let\endoldthebibliography=\endthebibliography
  \renewenvironment{thebibliography}[1]{%
    \begin{oldthebibliography}{#1}%
      \setlength{\parskip}{0.25ex}% 
      \setlength{\itemsep}{0.25ex}% 
  }%
  {% 
    \end{oldthebibliography}% 
  }



\usepackage{caption}
\captionsetup{labelsep=period}

\setlength{\parskip}{0.125\baselineskip}%
%\setlength{\parindent}{0pt}%

\input newcommands

%\raggedright


\begin{document}


%\setlength{\parindent}{0in} % Amount of indentation at the first line of a paragraph.


\pagestyle{fancy} 
\lhead{Approaching Exascale Models of Astrophysical Explosions} 
\rhead{PI: Zingale} \renewcommand{%
\headrulewidth}{0.0pt}

\begin{center}
\bf {PERSONNEL JUSTIFICATION AND MANAGEMENT PLAN} 
\end{center}


\section{PERSONNEL JUSTIFICATION}

%% What personnel are already in place and what are their roles on the
%% project? If applicable, describe personnel that will be hired on the
%% project in the future and their responsibilities. The INCITE program
%% welcomes proposals from individual PIs or teams of collaborators.

All personnel are in place.  Our team spans 4 institutions (Stony
Brook, Los Alamos National Laboratory, Lawrence Berkeley National
Laboratory, and Oak Ridge National Laborator).  Many of us have
collaborated together for more than 10 years on problems related to
those proposed here.  Furthermore, we (together will additional
collaborators at these same institutions) represent the principal
authors of three of the simulation codes used here, \maestro, \castro,
and \chimera.


\begin{tightitem}
\item {\bf Michael Zingale} will act as the lead for this project.  He
  will collaborate on all of the science applications, work on
  development and maintenance of the two simulation codes as needed
  for this project, and run simulations of the XRB.

\item {\bf Ann Almgren} will work on code development of both \maestro\
  and \castro\ in support of the proposed problems.  She will also be
  involved in the scientific analysis of the simulations proposed here.

\item {\bf John Bell} will be the main contact with the \boxlib\ group
  at LBNL.  We will work with John to incorporate performance
  improvements into \maestro\ and \castro.  John will also directly be
  involved in the scientific analysis of the simulations proposed
  here.

\item {\bf Alan Calder} will work primarily on the analysis and
  simulation design of the \castro\ white dwarf merger and
  black widow pulsar simulations.
  Alan will also be a lead on the ``extra'' exploratory applications
  proposed here, particularly the URCA calculations.

\item {\bf Brian Freisen} is a NSEAP-postdoc at NERSC who will aid us
  in optimizing \boxlib, in particular in porting the multigrid solver
  to GPUs.

\item {\bf Raph Hix}

\item {\bf Adam Jacobs} will be a postdoc at MSU at the time of this
  proposal, and will continue to work on the \maestro\ applications
  proposed here.  Adam is also the lead developer of the GPU version
  of the reaction networks in \maestro\ and \castro.

\item {\bf Dan Kasen} will perform post-processing radiation transfer
  calculations on the explosion results with \sedona, generating
  lightcurves and spectra.

\item {\bf Max Katz} will graduate with his PhD in a few months, and
  expects to continue his simulations on the \castro\ WDWD.  Max is
  also the lead developer of the GPU version of the equation of state.

\item {\bf Eric Lentz}

\item {\bf Chris Malone} will be the lead scientist on the XRB problem
  with \maestro\ (Chris did his PhD thesis on \maestro\ XRB simulations
  in 2D).  Chris will also assist in the reaction network development.

\item {\bf Bronson Messer}

\item {\bf Thomas Papatheodore}

\item {\bf Weiqun Zhang} is one of the lead \boxlib\ developers and will
  assist us in optimizing our codes for the OLCF machines, including the
  tiling efforts.

\end{tightitem} 

\section{MANAGEMENT PLAN}

We have been working together for over 10 years (you'll find many of
the names here on the list of publications under the current INCITE
award).  Although we identified leads for each of the science problems
above, we will all be involved in the science analysis and we expect
many of the CoIs here to be co-authors on the publications that result
from the proposed simulations.  Allocation of computing resources will
primarily follow the breakdown of time estimated in the narrative.  As
we work closely together and are in frequent contact, we do not expect
any tensions or issues with the distribution of this time.

For the code development, the SBU group travels to LBNL several times
a year to plan the development path of \maestro\ and \castro, and we
make frequent contact with each other through e-mail and phone
conferencing.  All of the codes are managed by the git version control
system and we use regular regression testing to ensure that the code
continues to perform as expected while undergoing development.  PI
Zingale will be the primary contact dealing with the management of the
award and preparing reports.  We expect that the graduate students
here will take a lead role in presenting the results to the community
at conferences, and to be the lead authors in the publications that
result.  All Co-Is will also communicate with one another regularly
using {\sf slack}, a web communication tool that integrates
conversations, github, and our test suites.

%% Describe the project's leadership team and how decisions are made to
%% allocate time to individuals or, for proposals with multiple
%% collaborating teams, subgroups within the project. Describe the focus
%% of each individual or subgroup. Successful proposals will include a
%% management plan that conveys to reviewers the interrelationship
%% between subgroups and how the sum of the parts will lead to scientific
%% discovery or engineering solutions that are the overarching goal of
%% the project. Also identify points of contact who will provide updates
%% on the status of the work including publications, awards, and
%% highlights of accomplishments.



\end{document}
