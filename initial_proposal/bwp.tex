A black widow pulsar (BWP) is a binary system consisting of a low mass
star and a pulsar, a rapidly rotating highly magnetized NS.  
The intense radiation from the pulsar interacts with the stellar companion,
ablating its outer layers, leaving behind a very low mass star that 
will eventually be completely disrupted 
\cite [See][and references therein]{bednareksitarek2013}. The interesting 
thing about 
BWPs is that the pulsars in these systems seem to be relatively massive, 
possibly due to accretion of substantial amounts of their companions. 

Uncertainties in the size and shape of the ablating companion, however, 
preclude an exact determination of the pulsar's mass. If the companion is 
point-like, the inferred mass is small.  If the companion is able to fill 
its Roche Lobe, allowing its mass to spill onto the pulsar, the inferred 
mass would be much larger. Our proposed study study will be the first 
to address modeling this problem with full 3D radiation hydrodynics
simulations to thereby better interpret observations. There have been 
no multi-D radiation hydrodynamical simulations 
of black widow pulsars that we are aware of, so the developments enabled 
here have the potential of being revolutionary and include removing
uncertainty in the inferred mass of the pulsar, an important step
in discerning the equation of state of dense matter.


